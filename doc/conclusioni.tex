\section{Conclusioni}

Come possibili sviluppi futuri del progetto proponiamo:
\begin{itemize}
    \item utilizzo di una tecnica diversa dall'image stitching per evitare rilevazioni ripetute degli stessi oggetti in frame diversi;
    \item re-implementazione sotto forma di azioni ROS delle diverse funzionalità che attualmente sono realizzate come servizi ROS, in modo da poter, ad esempio, avviare la detection quando Pepper inizia a riportare la testa nella posizione iniziale, ed eseguire le due operazioni contemporaneamente, così da ottimizzare il tempo totale di interazione;
    \item esecuzione del software utilizzando dispositivi di accelerazione per reti neurali, sia a bordo di Pepper, ad esempio utilizzando una delle schede Nvidia Jetson, che a distanza, ad esempio usando una GPU su di un server dedicato, al fine di misurarne le reali prestazioni in termini di tempo di esecuzione;
    \item valutazione delle performance in termini di rilevamento e classificazione su scene diverse.
\end{itemize} 